\chapter{混部场景导向的调度定制}\label{chap:sched_policy}

% 本章首先进行内核调度配置的定制,考虑网络敏感型应用要求内核调度以实时性为先,而离线CPU敏感型应用则要求内核调度以吞吐量为先,针对这两种需求设计了响应度优先与吞吐量优先两种内核调度配置。

% 对于CPU资源的隔离,常见的方式是显示地隔离不同应用能够使用的CPU,这种静态的划分方式并不能有效的使用资源,而在本章中提出了基于BPF调度器的调度队列隔离,高低优先级的应用在调度队列优先级机制之上共享相同CPU,使得高优先级应用运行时总是能够优先地使用CPU资源。

% 其次,相较于常见的调度类隔离,本章基于BPF调度器实现更为灵活,通过结合eBPF强大的监测能力,让BPF调度器感知高优先级应用CPU资源、网络资源的使用,而通过设置相关的资源约束,从而结合不同高优先级应用的资源使用特征与调度目标,更充分的利用资源。

为解决不同混部场景下任务调度机制的不足,本章主要展开混部场景导向的定制任务调度机制研究。首先,从内核任务调度的相关配置上,针对不同应用的需求,设计了响应度优先与吞吐量优先两种基础内核配置,随后,在Ext调度类的基础上研究场景导向的BPF Scheduler定制,设计了基于调度队列隔离的QoS保障策略,并在此基础上,结合eBPF的监测能力,进一步设计了CPU资源感知与网络资源感知的任务调度策略。

\section{内核调度配置定制}

Linux调度子系统提供了一些编译配置选项,用以针对场景进行优化,相关配置选项围绕时钟中断与抢占模型展开。

时钟中断是驱动Linux抢占式调度的核心机制,如表~\ref{tab:config_hz}所示,Linux内核提供了时钟中断频率与时钟中断处理模式两方面的配置。其中,中断的频率配置决定了调度滴答的周期,Linux提供了从100、250、300到1000四种不同的中断频率配置,一方面,越高的时钟中断频率意味着系统的整体响应度更好,另一方面,时钟中断频率决定了时间片的最小粒度,Linux在每个时钟中断中处理时间片记账,然而时钟中断间隔中存在任务切换的可能,容易导致时间记账的误差,而越高的时钟中断频率意味着越小的时钟中断间隔,间隔中发生任务切换的概率也会越小,因此越高的时钟中断频率意味着更准确的时间记账。

\begin{table}[H]
    \bicaption{\quad 内核时钟中断配置}{\quad Kernel Clock Interrupt Configuration}% caption
    \label{tab:config_hz}
    \footnotesize% fontsize
    \setlength{\tabcolsep}{4pt}% column separation
    \renewcommand{\arraystretch}{1.5}% row space 
    \centering
    \begin{tabular}{lc}
        \hline
        配置名称 & 描述 \\
        \hline
        HZ\_100  & 配置时钟中断频率为100  \\
        HZ\_250  & 配置时钟中断频率为250 \\
        HZ\_300  & 配置时钟中断频率为300 \\
        HZ\_1000 & 配置时钟中断频率为1000 \\
        HZ\_PERIODIC & 永远不要忽略时钟中断 \\
        NO\_HZ\_IDLE & 忽略空闲CPU上的时钟中断 \\
        NO\_HZ\_FULL & 忽略空闲CPU,以及只有一个可运行任务CPU上的时钟中断 \\
        \hline
    \end{tabular}
\end{table}

时钟中断会打断当前任务的执行,并产生上下文切换的开销,为此Linux内核提供了如表~\ref{tab:config_hz}所示的不同时钟中断处理模式,来减少特定场景中的不必要的时钟中断开销。在开启NO\_HZ选项之后,Linux内核允许指定部分CPU核心为NO\_HZ核心,而根据不同的时钟处理模式,这些CPU会在特定的场景下,如CPU空闲时,忽略时钟中断,以达到节能或减少对任务的干扰的目的。

抢占通常指高优先级任务打断低优先级任务的执行。在Linux内核中,中断产生时用户态任务总是会被抢占,但对于内核态代码,早期Linux内核在执行时会屏蔽中断,而在一些较长的内核代码处理路径中就容易导致系统整体的响应度下降。为此内核提供了如表~\ref{tab:config_preempt}所示的抢占模型,用于定制内核代码执行对于中断的处理模式,不同抢占模式的主要区别在于内核代码可被抢占位点的差异,而内核代码中可被抢占的位点越多,系统的实时性就会越好,但也同时也意味着任务的执行更容易被频繁地打断,因此需要根据不同应用的需要来进行选择。

\begin{table}
    \bicaption{\quad 内核抢占模式配置}{\quad Kernel Preemption Configuration}% caption
    \label{tab:config_preempt}
    \footnotesize% fontsize
    \setlength{\tabcolsep}{4pt}% column separation
    \renewcommand{\arraystretch}{1.5}% row space 
    \centering
    \begin{tabular}{lc}
        \hline
        配置名称 & 描述 \\
        \hline
        PREEMPT\_NONE  & 内核代码保持执行直到主动放弃CPU  \\
        PREEMPT\_VOLUNTARY  & 开启了内核代码中的抢占位点 \\
        PREEMPT  & 提供更多的内核代码抢占位点,实现完全抢占 \\
        PREEMPT\_RT & 进一步修改内核代码的实现,如锁机制,实现实时可抢占性 \\
        \hline
    \end{tabular}
\end{table}

时钟中断和抢占模式配置共同决定了Linux调度子系统的整体响应度,根据应用的不同需求,本研究设计了响应度优先与吞吐量优先两种基本调度子系统内核配置。其中,响应度优先配置采用更高的HZ与激进的抢占模式来提升整体响应度,满足应用对于延迟与实时性的需求。吞吐量优先配置则采用较低的HZ与保守的抢占模式,让CPU在有限的时间内更多的执行任务而不是中断处理,来满足应用对于局部性和性能的需求。

\section{BPF任务调度策略框架}

% Linux Fair调度队列问题
% - 单个CPU上的任务同属于一个调度队列
% - 优先级机制用于限制不同任务对CPU时间的使用,并不能保证高优先级的应用一定能够比低优先级应用优先调度
% 使用实时调度类与Idle调度类存在的问题
% - 运行在实时调度类中的LC任务如配置不当,一方面不利于应用性能,另一方面容易造成系统崩溃
% - 将低优先级的任务允许在Idle类是嵌入式领域的常用做法,但不够灵活,且违背Linux设计语义

混部场景中使用优先级机制来保障高优先级应用QoS时常用的手段,然而Linux内核提供的优先级机制却存在许多不足。对于调度类优先级,将任务设置为较高优先级的调度类,能够保证高优先级应用就绪时抢占低优先级应用,然而这种方式存在一些问题,首先,将高优先级任务设置为高优先级调度类时,需要充分考虑应用的特点与高优先级调度类的配置,而不恰当的配置一方面会影响应用本身的性能,另一方面也有可能引发整个系统的阻塞,其次,将低优先级任务设置为低优先级调度调度类,如Idle调度类时,一方面,Idle调度灵活性有限,不能很好地满足任务需求,另一方面,将常规任务放置到Idle调度类破坏了Idle调度类的设计语义。

对于调度队列中的优先级,以默认的Fair调度类为例,在CFS调度器的运行过程中,高优先级任务的vruntime相对增加得缓慢,因此能够得到更多的调度机会,然而随高优先级vruntime的不断累积,CFS为考虑调度的公平性存在抢占高优先级应用的可能,同时,CFS调度器启发式逻辑对于新创建与刚唤醒应用的特殊处理,也会导致高优先级应用被抢占。因此,Fair调度类很难在混部场景中持续地保证高优先级应用的QoS。其次, 调度队列优先级无法跨调度队列地产生作用,一方面,高优先级应用会与运行在其他CPU上的低优先级应用争抢资源,另一方面,优先级的相对性也导致高优先级应用在迁移到其他CPU、其他机器上时,有限性也并不能完全保证。

而为解决混部场景下Linux调度机制的问题,本研究设计实现了BPF任务调度策略框架,首先,Ext调度类优先级处于Fair调度类与Idle调度类之间,因此能够实现运行队列的隔离,从而保证高优先级应用就绪时总是能够抢占低优先级应用,其次,相较于其他高优先级调度类,Ext调度类能够通过自定义BPF Scheduler实现更灵活的调度策略,来满足不同应用的调度需求,最后,本研究实现了集中式BPF调度,任务在一个全局的队列中进行调度,从而能够实现全局的优先级保证。

% 集中式策略与分布式策略图

为适应不同混部应用的调度需求,BPF Scheduler提供了三种基本的调度策略实现:

\begin{enumerate}
    \item 基于FIFO的非抢占式调度:BPF Scheduler中使用FIFO DSQ(分发队列),并设置每个入队任务的时间片为无限,任务运行直到主动退出,随后BPF Scheduler取出下一个任务执行。
    \item 基于RR的抢占式调度:BPF Scheduler中使用FIFO DSQ, 并按照配置设置每个入队任务的时间片,每次时钟滴答时检查当前任务的剩余时间片,抢占时间片耗尽的任务并切换到下一个任务执行。
    \item 基于vtime的抢占式调度:BPF Scheduler中使用vtime DSQ,vtime DSQ按照vtime大小对任务进行排序,初始时设置每个入队任务的初始vtime,并在每个时钟滴答里根据任务的静态优先级更新vtime,抢占发生时,vtime越小的任务越优先被调度。
\end{enumerate}

\section{资源感知的BPF任务调度策略}

% 对于CPU资源的隔离,常见的方式是显示地隔离不同应用能够使用的CPU,这种静态的划分方式并不能有效的使用资源,而在本章中提出了基于BPF调度器的调度队列隔离,高低优先级的应用在调度队列优先级机制之上共享相同CPU,使得高优先级应用运行时总是能够优先地使用CPU资源。

% 其次,相较于常见的调度类隔离,本章基于BPF调度器实现更为灵活,通过结合eBPF强大的监测能力,让BPF调度器感知高优先级应用CPU资源、网络资源的使用,而通过设置相关的资源约束,从而结合不同高优先级应用的资源使用特征与调度目标,更充分的利用资源。

\subsection{CPU资源感知策略} 

% 与 Core Scheduler的区别
% - Core Schedule使用固定的CPU拓扑
% - Bpf Schedule可自由定义CPU Cgroup

Linux采用调度域机制来描述不同的CPU特性,而在第一章的分析中提到,由于SMT、NUMA、CPU Turbo等技术的存在,独立的CPU之间仍然存在资源的竞争,这就导致了混部场景下的吵闹邻居现象。单一调度队列中,优先级机制能够指示任务的调度行为,但在不同CPU调度队列上,优先级机就几乎不能产生影响,如在一个SMT系统中,即便运行在不同Sibling上的任务存在优先级差异,但此时调度器并不能保证高优先级任务的资源使用。

对于CPU敏感型应用,如上问题会引发较大的性能劣化,为此本研究进一步设计了CPU资源感知策略。策略允许低优先级任务的调度的CPU阈值,允许当低优先级任务的可用CPU数量达到阈值时,才能进行任务调度,否则则需要让出CPU资源。相较于Linux Core调度,CPU资源感知策略不局限于SMT编组,提供更灵活的调度方案。而相较于Cgroup中的cpu set机制,CPU资源感知策略使用更动态的CPU数量而不是固定的CPU亲和性,从而能够实现高优先级任务空闲时,利用所有的空闲核心,而当高优先级任务运行时,来根据阈值控制并发数量。

\begin{algorithm}
    \caption{Pseudocode for Enhanced Task Scheduling Isolation Mechanism}
    \label{alg:cpu_aware_sched}
    \begin{algorithmic}[1]
    
    \Function{cpuAcquire} {cpu}
    \State Insert cpu to CPU\_Map;
    \EndFunction
    
    \Function{cpuRelease} {cpu}
    \State Remove cpu from CPU\_Map;
    \EndFunction
    
    \Function {schedule}{}
        \While{True}
            \If{$\text{CPU\_Acquired + CPU\_Idle} < \text{CPU\_Available}$ }
                \State Set Exclusive Flag;
                \For{each no\_hz core C}
                    \State Send an IPI to envoke target CPU rescheduling.
                \EndFor
                \State Yield to higher priority scheduler class;
            \EndIf
            \State Remove Exclusive Flag;
            \State Dispatched tasks;
        \EndWhile
    \EndFunction
    \end{algorithmic}
\end{algorithm}

而在实际BPF Scheduler中,选择使用一个全局DSQ,并通过cpu\_acquire,cpu\_release两个回调函数来动态获取当前Ext调度类中所能管理的CPU数量,而具体的调度逻辑如伪代码~\ref{alg:cpu_aware_sched}所示,首先BPF Scheduler会依据不同调度类CPU的数量,来判断是否有更高优先级调度类中的任务正在运行,一旦发现,则首先设置一个Exclusive Flag, 用于在其他回调中共享状态,其次,对于处于NoHZ状态下的CPU,发送IPI以注入一次重调度,而当系统中没有更高优先级调度类的任务运行时,则正常进行任务的分发。

\subsection{网络资源感知策略}

% LC应用负载与Epoll的关系
% - 原理
% - 流量,epoll wait图,相关性
% 网络资源感知策略

网络资源敏感型应用的构成中并非所有部分都与网络相关,如Redis主线程监听网络请求并进行处理,而用于快照的辅助线程则与网络无关,同时,在第三章的应用画像中也发现网络资源敏感型应用的资源敏感度随负载而变化。以上都使得网络资源的使用情况能够作为一种进行调度决策的因素。

大多数高性能网络应用都基于epoll机制实现。epoll机制允许进程同时监听多个连接,并在连接就绪时进行批量处理,相较于阻塞型网络系统调用,使用epoll能够在大量请求时减少频繁地阻塞唤醒过程,从而支持更高的并发量。epoll机制涉及到多个系统调用,而其中最为核心的是epoll\_wait系统调用,进程在注册等待事件之后,就能够通过epoll\_wait系统调用来获取当前的就绪事件,当存在就绪事件时,epoll\_wait系统调用会返回就绪的事件数量,而当没有事件就绪时,epoll\_wait仍然会阻塞进程。

\begin{figure}[H]
    \centering
    \begin{subfigure}[b]{0.49\textwidth}
        \includegraphics[width=\textwidth]{epoll_memcached}
        \caption{Memcached请求数量与Epoll Wait事件数量}
        \label{fig:epoll_memcached}
    \end{subfigure}
    \begin{subfigure}[b]{0.49\textwidth}
        \includegraphics[width=\textwidth]{epoll_redis}
        \caption{Redis请求数量与Epoll Wait事件数量}
        \label{fig:epoll_redis}
    \end{subfigure}
\bicaption{\quad LC应用负载与Epoll事件}{\quad Requests and Epoll Events on LC}
\label{fig:epoll_request}
\end{figure}

综合以上分析,网络资源的使用监测可通过监听epoll\_wait系统调用的执行以及其返回参数实现,而使用eBPF技术能够方便地对系统调用进行探测。结合第三章中的eBPF系统调用探测的实现,网络资源监听eBPF程序在epoll\_wait返回处进行插桩,并统计epoll\_wait的执行频率与总事件计数,其次,eBPF程序额外记录了当次epoll\_wait完成的时间戳及事件数量,用以捕获突发的事件。截取Redis与Memcached在工作负载下的指标序列,绘制如图~\ref{fig:epoll_request}所示的折线图,能够分析,对于Redis与Memcached这两类典型的网络应用,epoll\_wait事件数量与请求数量高度相关,由于实验中由于请求计量在client侧,epoll\_wait事件计数则在server侧,因此两段序列会存在一定的偏移。

eBPF程序之间的数据能够方便地通过BPF Map共享,而基于对于epoll wait的探测程序实现的网络资源感知BPF Scheduler具体逻辑如伪代码~\ref{alg:network_aware_sched}所示。在BPF Scheduler在获取到CPU的管理权限之后,还需要根据当前系统中的网络资源使用情况来判断是否进行低优先级任务的调度,网络资源的阈值可以结合第三章中的无干扰画像进行灵活的配置,主要在低负载段允许任务的并发以充分利用系统资源。

\begin{algorithm}[H]
    \caption{Pseudocode for Network Resource Constraints Scheduling Strategy}
    \label{alg:network_aware_sched}
    \begin{algorithmic}[1]

    \Function {do\_epoll\_wait\_return}{$nr\_event$}
        \State Write timestamp of current epoll wait;
        \State Write $nr\_event$ of current epoll wait;
        \State Update total event count;
        \State Update count of epoll wait;
    \EndFunction

    \Function {schedule}{}
        \While{True}
            \State Read epoll wait delta from bpf map;
            \If{$\text{EPOLL\_WAIT\_DELTA} < \text{EPOLL\_WAIT\_THRESHOLD}$ }
                \State Set Exclusive Flag;
                \For{each no\_hz core C}
                    \State Send an IPI to envoke a rescheduling;
                \EndFor
                \State Yield to higher priority scheduler class;
            \EndIf
            \State Remove Exclusive Flag;
            \State Dispatched tasks;
        \EndWhile
    \EndFunction
    \end{algorithmic}
\end{algorithm}

基于调度队列的隔离会在高优先级任务执行时进行避让,因此BPF Scheduler网络资源感知策略仍然相对保守,导致无法充分地使用系统资源。对此有两种解决方式,第一种方式是将高优先级应用也加入到Ext调度类中,并单独为其分配一个DSQ,这种方式对BPF Scheduler的实现要求较高。而第二种方式则相对简单,eBPF的优势在于其灵活性,因此上述逻辑也可以变更为在低负载时,关闭BPF Scheduler,而根据第二章的技术分析,此时所有的EXT调度类任务将回退到Fair调度类中,从而解除调度队列隔离,而当负载达到一定阈值时,再使能BPF Scheduler来保障高优先级任务的性能。

% \subsection{基于PMU实现的其他资源感知策略}

\section{实验设计与分析}

\subsection{响应度优先内核性能}

% redis \ memcached

Control Zone响应度优先配置使用HZ\_1000配置时钟中断,并开启PREEMPT抢占模式。响应度优先内核的优势主要体现在多任务调度场景,因此在实验设计上,选择Redis、Memcached两种CPU敏感型应用来分别与CPU干扰应用混部以构造多任务场景,并使用CloudHypervisor默认提供的内核作为基准进行比较。实验在一个4CPU 512MB内存的Control Zone中进行。

\begin{figure}[H]
    \centering
    \begin{subfigure}[b]{0.49\textwidth}
      \includegraphics[width=\textwidth]{redis_response}
      \caption{Redis与干扰混部}
      \label{fig:redis_response}
    \end{subfigure}
    \begin{subfigure}[b]{0.49\textwidth}
      \includegraphics[width=\textwidth]{memcached_response}
      \caption{Memcached与干扰混部}
      \label{fig:memcached_response}
    \end{subfigure}
    \bicaption{\quad 混部场景下的响应度优先内核}{\quad Response Priority Kernel in Mixed Deployment Scenarios}
    \label{fig:lc_response}
\end{figure}

混部实验结果如图~\ref{fig:lc_response}所示,在启用Control Zone响应度优先内核后,无论是Redis还是Memcached,在P99.9延迟上都要优于CloudHypervisor的默认内核,Reponse内核使用了更高的时钟中断频率,因此能够更快地在延时敏感应用与干扰应用中进行切换,同时在PREEMPT抢占模式下,网络中断能够更及时地进行处理,降低请求处理链路的整体延时。

\subsection{吞吐量优先内核性能}

% graph500(time) \ ffmjpeg 

Control Zone吞吐量优先配置使用HZ\_100配置时钟中断,并开启PREEMPT\_NONE抢占模式。吞吐量优先内核能够在任务量较少的场景中,让任务保持CPU资源的占用,从而更好地利用局部性,因此实验使用单一任务场景,选择Graph500作为目标应用,并对比其在CloudHypervisor默认内核、Throughput内核以及Response内核下的运行情况。实验在一个4 CPU、512MB内存的Control Zone中进行。

\begin{figure}[H]
    \centering
    \includegraphics[width=0.5\textwidth]{avg_graph500_runtime}
    \bicaption{\quad 吞吐量优先内核配置优化效果}{\quad Throughput Discrepancy Across Different Configurations} 
    \label{fig:avg_graph500_runtime}
\end{figure}

Graph500主要执行图计算算法,因此执行时间是其重要的性能指标,实验中使用time工具记录任务的执行时间,包括用户态与内核态的执行时间, 具体实验结果如图~\ref{fig:avg_graph500_runtime}所示,其中使用Throughput内核的Control Zone消耗的时间最短,而使用Response内核的Control Zone则消耗的最长的时间,分析执行时间占比,在Throughput内核下,任务运行过程中内核态时间消耗几乎没有,而在Response内核中,内核态时间则占用了较大比例,造成这一结果的主要原因是时钟中断与NO\_HZ配置,越高的时钟中断频率会引发越导致越频繁的陷入内核态,一方面使得内核态时间变得更长,另一方面也会导致局部性的破坏导致任务执行速度的减慢。

\subsection{CPU资源感知策略效果}

% 普通场景
% 硬件场景:SMT

CPU感知BPF调度策略实验首先验证调度策略对高优先级任务的QoS保障能力,实验在一个4 CPU、1024MB内存的虚拟机中展开,其中高优先应用为Mysql与Memcached,低优先级应用为CPU干扰应用。混部结果如图~\ref{lc_bpf_sched}所示,总体来看,CPU资源感知策略能够使LC应用达到与未受干扰情况下相近的性能,其中如图~\ref{}所示,MySql性能最高提升了xx,同时如图~\ref{}所示,于Memcached延迟也最高降低了xx,这一方面得益于Ext调度类在调度队列上的隔离性,使得高优先应用总是能够及时地得到调度,同时,通过CPU资源感知,进一步让其他核心上的低优先级任务感知高优先级任务的运行,并根据阈值来判断是否要出让CPU以避免干扰。

\begin{figure}[H]
    \centering
    \begin{subfigure}[b]{0.49\textwidth}
        \includegraphics[width=\textwidth]{mysql_perf}
        \bicaption{\quad Mysql与干扰混部}{\quad MySQL with Interference} 
        \label{fig:mysql_perf}
    \end{subfigure}
    \begin{subfigure}[b]{0.49\textwidth}
        \includegraphics[width=\textwidth]{bpf_sched_memcached}
        \bicaption{\quad Memcached与干扰混部}{\quad Memcached with Interference} 
        \label{fig:bpf_sched_memcached}
    \end{subfigure}
\bicaption{\quad CPU感知调度下的LC应用QoS保障}{\quad LC Application Latency Stability}
\label{fig:lc_bpf_sched}
\end{figure}

对于SMT场景,由于兄弟核心之间存在片上共享资源,CPU资源感知调度对于这一场景下的QoS保障更为重要,实验在一个2 CPU、1024MB内存的虚拟机中展开,其中CPU配置为兄弟核,同时内核不关闭Core调度机制。实验结果如图~\ref{fig:redis_smt}所示,相较于EEVDF调度器,CPU资源感知BPF调度器在SMT场景中, 能够将Redis延迟最高降低79.6\%,并且最低也有33.7\%的延迟降低效果,同时从图~\ref{fig:lc_box}中能够看出,在CPU资源感知BPF调度器下,Redis的延迟稳定性也得到了一定的保障,这主要是因为CPU资源感知BPF调度器完全运行在内核态中,能够快速的决策并调度任务,从而避免了许多用户态调度器存在的“乒乓效应”。

\begin{figure}[!htbp]
    \centering
    \includegraphics[width=0.6\textwidth]{redis_smt}
    \bicaption{\quad SMT下Redis与干扰混部}{\quad Redis with Interference On SMT} 
    \label{fig:redis_smt}
\end{figure}

\begin{figure}[H]
    \centering
    \begin{subfigure}[b]{0.49\textwidth}
        \includegraphics[width=\textwidth]{cpu_aware_box_bpf_sched}
        \caption{无干扰下Redis延迟}
        \label{fig:cpu_aware_box_bpf_sched}
    \end{subfigure}
    \begin{subfigure}[b]{0.49\textwidth}
        \includegraphics[width=\textwidth]{cpu_aware_box_no_interference}
        \caption{混部场景BPF调度器}
        \label{fig:cpu_aware_box_no_interference}
    \end{subfigure}
    \begin{subfigure}[b]{0.49\textwidth}
        \includegraphics[width=\textwidth]{cpu_aware_box_eevdf}
        \caption{混部场景EEVDF调度器}
        \label{fig:cpu_aware_box_eevdf}
    \end{subfigure}
    \begin{subfigure}[b]{0.49\textwidth}
        \includegraphics[width=\textwidth]{cpu_aware_box_eevdf_nice}
        \caption{混部场景EEVDF高优先级}
        \label{fig:cpu_aware_box_eevdf_nice}
    \end{subfigure}
\bicaption{\quad LC应用延迟稳定性}{\quad LC Application Latency Stability}
\label{fig:lc_box}
\end{figure}


% \subsection{网络资源感知策略}

% \subsection{内存资源感知策略}
% Redis

\section{本章小结}

本章主要首先针对画像分析中不同类型应用,基于内核中关于HZ与抢占模型的配置,设计了调度配置的定制优化,提供了吞吐量优先与响应度优先两种不同调度目标的内核配置。

随后设计了混部场景下高优先应用与低优先级应用基于BPF调度策略的QoS保障机制,首先,提出了较为粗粒度的BPF调度队列隔离策略,利用调度类的优先级特性,使得在单个CPU上,低优先级能够感知高优先级任务的执行并出让CPU资源,从而保障高优先级应用的执行。

其次,基于BPF调度队列隔离策略,结合eBPF对于CPU与网络资源的感知实现更灵活的BPF资源感知调度策略,一方面相较于静态的CPU亲和性配置,CPU资源感知的调度策略能够动态地利用高优先级任务没有用到的核心,另一方面利用eBPF程序之间的交互性,能够方便地将其他子系统中的监测信息传递给BPF Scheduler进行辅助决策。

最后,比较了不同内核下的应用性能,说明对于不同的应用而言,定制的内核能够在一定程度上提升其运行性能。最后,通过不同应用,不同硬件环境下的混部实验,说明了使用BPF Scheduler来针对不同的混部场景进行定制优化,在提升高优先级应用的性能保障效果上的优势。