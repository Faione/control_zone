\chapter{实验设计与结果分析}\label{chap:exp}

% 说明实验环境
% - 硬件环境
% - 软件环境
% - 使用偏好(Master Node)
% 软硬件说明
% - redis、mysql
% 基准测试
% - 开销分析
% 不同内核配置实验
% 隔离性实验
%  - 资源限制效果说明
% 调度策略实验
%  - 互斥调度
% 

\section{概述}

\section{实验环境}

研究中使用两台服务作为实验环境,服务硬件信息如~\ref{exp_env}表所示。计算资源上,服务器CPU支持Intel RDT,具备cdp与mba特性,能够监测LLC与Memory Bandwidth且支持LLC路数限制。网络资源上,服务器网卡支持SRIOV,对网络性能有要求的虚拟机能够利用硬件直通来实现高性能网络。

\begin{table}
    \bicaption{\quad 服务硬件参数}{\quad Server Hardware Information}% caption
    \label{tab:exp_env}
    \footnotesize% fontsize
    \setlength{\tabcolsep}{4pt}% column separation
    \renewcommand{\arraystretch}{1.5}% row space 
    \centering
    \begin{tabular}{lcc}
        \hline
        CPU & Intel Xeon Gold 6148 (40cores) * 2 \\
        Processor Core Frequency & 2.4GHz,Turbo 3.7 GHz \\
        L1 Caches & 32KB * 40,  8-way set associative, split D/I \\
        L2 Caches & 1024KB * 40, 16-way set associative \\
        L3 Caches & 28160KB, 11-way set associative \\
        Main Memory & 32GB * 8, 2666MHz DDR4 \\
        NIC & Intel Corporation Ethernet Connection X722 for 10GbE SFP+(10Gbit) \\
        \hline
    \end{tabular}
\end{table}

