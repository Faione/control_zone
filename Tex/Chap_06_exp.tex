\chapter{沙箱实验与结果分析}\label{chap:exp}

% 说明实验环境
% - 硬件环境
% - 软件环境
% - 使用偏好(Master Node)
% 软硬件说明
% - redis、mysql
% 基准测试
% - 开销分析
% 不同内核配置实验
% 隔离性实验
%  - 资源限制效果说明
% 调度策略实验
%  - 互斥调度
% 

\section{概述}

\section{实验环境}

实验环境有两台服务器构成,服务器硬件信息如表~\ref{tab:exp_env}所示。在CPU资源上,每台服务器上包含有两个Socket,单台总计80个物理核心,划分为4个Numa Node,同时,CPU均开启超线程,并使能Intel RDT,从而为可观测性基础设施提供末级缓存及内存带宽的监控,并为虚拟机提供按路数的末级缓存划分和固定补偿的内存带宽调控功能。在网络资源上,服务网卡支持SRIOV技术,能够为有网络性能需求的虚拟机提供硬件直通服务。

\begin{table}
    \bicaption{\quad 服务器硬件参数}{\quad Server Hardware Information}% caption
    \label{tab:exp_env}
    \footnotesize% fontsize
    \setlength{\tabcolsep}{4pt}% column separation
    \renewcommand{\arraystretch}{1.5}% row space 
    \centering
    \begin{tabular}{lc}
        \hline
        硬件资源 & 硬件信息 \\
        \hline
        CPU & Intel Xeon Gold 6148 (40 cores) * 2 \\
        Processor Core Frequency & 2.4GHz,Turbo 3.7 GHz \\
        L1 Caches & 32KB * 40,  8-way set associative, split D/I \\
        L2 Caches & 1024KB * 40, 16-way set associative \\
        L3 Caches & 28160KB, 11-way set associative \\
        Main Memory & 32GB * 8, 2666MHz DDR4 \\
        NIC & Intel Corporation Ethernet Connection X722 for 10GbE SFP+(10Gbit) \\
        \hline
    \end{tabular}
\end{table}

每台服务器的系统软件环境如表~\ref{tab:system_env}所示。在操作系统上,实验中选择使用较常见的Ubuntu22.04 LTS,Ubuntu同时也是Sched Ext优先支持的发行版,能够较方便地通过包管理工具安装预编译的Sched Ext内核。在虚拟换运行时上,Qemu采用发行版所支持的稳定版,而以轻量为目标的CloudHyeprvirsor则采用自编译的最新发布版本。

\begin{table}
    \bicaption{\quad 服务器系统环境}{\quad Server System Information}% caption
    \label{tab:system_env}
    \footnotesize% fontsize
    \setlength{\tabcolsep}{4pt}% column separation
    \renewcommand{\arraystretch}{1.5}% row space 
    \centering
    \begin{tabular}{lc}
        \hline
        软件类型 & 软件信息 \\
        \hline
        系统 & Ubuntu 22.04.3 LTS  \\
        内核 & 5.15.0-79-generic \\
        虚拟化运行时 & cloud-hypervisor v38.0-150 \\
                   & QEMU emulator version 6.2.0 \\
        其他        & libvirtd 8.0.0 \\
        \hline
    \end{tabular}
\end{table}

除系统软件之外,每台服务器还按照需要部署了其他关键服务。其中可观察基础设施按照第三章中所论述的架构进行搭建,Master节点上部署的了Prometheus与Grafana,用于进行数据采集与离线分析,Node节点上则部署的第三章中所提到的一系列Exporter,提供各个维度数据的采集能力。Master除对数据进行采集、存储、分析外,还额外部署了Harbor来对外提供容器管理服务,并承担大部分的配置文件存储。Node作为主要的实验场地,安装了Control Zone沙箱的所有相关的组件,并承担主要的服务运行。实验中对于Client-Server类型的任务,为尽可能地模拟真实环境,因此一般将Client放置在Master上。最后,实验中所涉及的关键服务都以容器镜像的形式分发并运行,因此在每个服务器上都需要安装容器运行时,而在容器运行时的选择上,对性能不敏感而对稳定性有要求的Master上使用Docker来提供容器服务,而在Node上,则使用Podman作为容器运行时,Podman相较Docker更加轻量,同时不存在Docker、Containerd等后台驻留服务,能够提供较为纯净的容器运行环境。

\section{内核调度配置实验}

\subsection{时钟与抢占模型}

Linux调度子系统在设计时力图覆盖足够广泛的场景,但同时也提供了一些编译配置选项,用以针对场景进行优化,这些配置选项围绕时钟与抢占模型的设置展开。

时钟与Linux调度子系统密切相关,在第二章论述中提到,时钟中断是驱动Linux抢占式调度的核心机制,中断的频率决定了调度滴答的周期,而更高的时钟中断频率意味着系统的整体响应度越好。针对时钟中断,内核主要提供了两方面的配置,如表~\ref{tab:config_hz}所示,其中最直接的就是时钟中断的频率配置,内核提供了从100、250、300到1000四种不同的中断频率配置,以满足不同的场景下对于响应度的需求,而值得注意的是,由于内核在每个时钟中断处理中进行时间记账,并将两次时钟中断所间隔的时间片累计到当前进程的记账中,然而在实际情况下两次时钟中断中间很有可能夹杂了其他任务的处理,所以更高的时钟中断频率不仅能带来响应度的提升,还能够增强任务时间记账的准确性。其次,考虑到时钟中断会引入额外的计算开销,而在一些场景中这些开销是非必要的,如当CPU进入Idle状态,此时处理时钟中断不仅没有意义,还会增加系统的开销,因此内核提供了NO_HZ相关配置来在特定场景下屏蔽时钟中断。

\begin{table}
    \bicaption{\quad 内核时钟中断配置}{\quad Kernel Clock Interrupt Configuration}% caption
    \label{tab:config_hz}
    \footnotesize% fontsize
    \setlength{\tabcolsep}{4pt}% column separation
    \renewcommand{\arraystretch}{1.5}% row space 
    \centering
    \begin{tabular}{lc}
        \hline
        配置名称 & 描述 \\
        \hline
        HZ_100  & 配置时钟中断频率为100  \\
        HZ_250  & 配置时钟中断频率为250 \\
        HZ_300  & 配置时钟中断频率为300 \\
        HZ_1000 & 配置时钟中断频率为1000 \\
        HZ_PERIODIC & 永远不要忽略时钟中断 \\
        NO_HZ_IDLE & 忽略空闲CPU上的时钟中断 \\
        NO_HZ_FULL & 忽略空闲CPU,以及只有一个可运行任务CPU上的时钟中断 \\
        \hline
    \end{tabular}
\end{table}

抢占指执行任务时,允许高优先级的任务打断低优先级任务的执行,从而提供更好的响应度。用户态任务的执行总是能够被打断,而内核态任务的抢占则较为复杂,为此内核提供了抢占模型的编译配置,允许用户针对不同场景进行调整,相关配置如表~\ref{tab:config_preempt}所示, 内核提供了PREEMPT_NONE、PREEMPT_VOLUNTARY、PREEMPT以及PREEMPT_RT四种抢占模式,四种模式下内核的响应度逐步增强。

\begin{table}
    \bicaption{\quad 内核抢占模式配置}{\quad Kernel Preemption Configuration}% caption
    \label{tab:config_preempt}
    \footnotesize% fontsize
    \setlength{\tabcolsep}{4pt}% column separation
    \renewcommand{\arraystretch}{1.5}% row space 
    \centering
    \begin{tabular}{lc}
        \hline
        配置名称 & 描述 \\
        \hline
        PREEMPT_NONE  & 内核代码保持执行直到主动放弃CPU  \\
        PREEMPT_VOLUNTARY  & 开启了内核代码中的抢占位点 \\
        PREEMPT  & 提供更多的内核代码抢占位点,实现完全抢占 \\
        PREEMPT_RT & 进一步修改内核代码的实现,如锁机制,实现实时可抢占性 \\
        \hline
    \end{tabular}
\end{table}

时钟和抢占模式的配置能够决定系统的响应度,但实际的配置需要考虑使用的场景,如更高的HZ虽然能够提升系统的整体响应度,但由于时钟中断的增多,CPU在于有限的时间片内处理实际任务的时间就会变少,同时频繁的上下文切换也破坏了程序的时间局部性,从而影响到系统的整体吞吐。通常而言,较高的HZ与激进的抢占模式有利于降低延时,而较低的HZ与保守的抢占模式有利于系统的整体吞吐。对此Control Zone提供了两种预编译的内核,用于处理响应度优先与吞吐量优先两种场景。


\subsection{响应度优先}

% redis \ keydb

Control Zone响应度优先配置使用HZ_1000配置时钟中断,并开启PREEMPT抢占模式,并选取redis、keydb两种LC应用进行实验。


\subsection{吞吐量优先}

% graph500(time) \ ffmjpeg 

Control Zone吞吐量优先配置使用HZ_100配置时钟中断,并开启PREEMPT_NONE抢占模式,并选取graph500、ffmjpeg两种BE应用进行实验。实验结果如图~\ref{fig:avg_graph500_runtime}所示

\begin{figure}[!htbp]
    \centering
    \includegraphics[width=0.8\textwidth]{avg_graph500_runtime}
    \bicaption{\quad 不同配置下的吞吐量差异}{\quad Throughput Discrepancy Across Different Configurations} 
    \label{fig:avg_graph500_runtime}
\end{figure}


\section{BPF调度策略实验}

\subsection{保守调度策略}

Mysql

\begin{figure}[!htbp]
    \centering
    \includegraphics[width=0.8\textwidth]{mysql_perf}
    \bicaption{\quad 保守调度策略效果}{\quad Effectiveness of Conservative Scheduling Policies} 
    \label{fig:mysql_perf}
\end{figure}

% \subsection{内存资源感知策略}
% Redis

% \subsection{网络资源感知策略}

\section{本章小结}