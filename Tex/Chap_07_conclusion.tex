\chapter{总结与展望}\label{chap:theories_tech}

% TODO: 优化表达

\section{工作总结}

混部技术是数据中心提升资源利用率的主要手段,但也引发了资源竞争导致应用QoS劣化的问题。本文为解决混部场景下的应用QoS保障问题,首先设计了一套面向虚拟机的可观测性系统,并结合与云厂商的合作经验,选取典型应用进行画像分析。随后,本文设计实现了Control Tower内核调度框架,并展开了针对混部场景的任务调度定制。最后,为解决单点上无法同时允许多种调度机制的问题,本文设计实现了Control Zone,一种面向混部场景的沙箱,支持混部应用与内核调度器的协同部署,从而更好地进行应用QoS保障。具体贡献如下:

\begin{enumerate}
    \item 云场景典型应用监测与画像分析。本文选取云场景中的7种典型应用,并使用KVM虚拟机作为运行环境,分析应用在资源使用倾向与敏感度上的差异。首先,本文基于Prometheus生态设计实现了一套面向KVM虚拟机的可观测性系统,能够从Host、Hypervisor与App等多重维度采集的丰富的指标数据,并开创性地使用eBPF技术探测虚拟机在系统调用、后端设备模拟上的性能指标。随后,本文开展了基准性能实验与性能劣化实验,并通过华为云竖亥Benchmark模拟真实的线上负载。最后,基于从可观测性系统获取的7种典型应用各项性能指标,展开画像分析并给出结论。

    \item 混部场景导向的任务调度定制研究。本文基于画像分析的结论,展开混部场景导向的任务调度定制研究。首先针对应用对于延迟与吞吐量的不同需求,实现了响应度优先与吞吐量优先两种内核配置。随后在Control Tower调度框架的基础上,针对不同混部场景量身定制任务调度机制。本文实现了CPU资源感知与网络资源感知两种Control Tower调度策略,分别能够在SMT硬件场景与LC应用动态负载下更好保障应用的QoS。

    \item 设计实现了一种调度动态可定制沙箱。本文基于KVM虚拟化实现了Control Zone,一种运行时调度可变的沙箱机制。Control Zone提供了一种在单个物理机上同时运行多种不同调度机制的方案。Control Zone允许将混部应用与内核调度策略进行协同部署,能够在特定混部场景下更好地保障应用的QoS。同时,Control Zone支持在运行时改变任务调度策略,从而适应混部场景的软硬件环境变化。最后,Control Zone还支持丰富的资源隔离手段,避免不同KVM虚拟机间的相互干扰。

\end{enumerate}

本文的实验分别在云厂商真实环境与两台实际物理机上展开。在云厂商的真实环境中主要展开典型应用画像实验,使用两台华为云C6实例与华为云竖亥Benchmark来尽量模拟真实的线上业务环境。两台物理机上的实验主要针对Control Tower调度策略与Control Zone沙箱性能展开,分析了不同内核配置、不同Control Tower调度策略对于特定混部场景的QoS保障效果,以及使用轻量化虚拟机运行时、精简内核的Control Zone在性能上的开销。

本文设计实现的可观测性系统、Control Tower任务调度框架与Control Zone调度动态可定制沙箱提供了一种基于内核调度的混部场景QoS保障方案。首先,基于可观测性系统丰富的指标数据,能够了解应用的性能状况。随后,允许针对不同的混部场景量身定制Control Tower任务调度策略,来实现更好的应用QoS保障。最后,Control Zone沙箱提供了混部方案的部署与管理手段,同时能够利用可观测性系统验证混部方案的效果并持续迭代。

\section{未来展望}

本文只设计了CPU与网络两种资源感知的调度策略,而实际上eBPF技术能够提供丰富的内核层次可观测性,借助对驱动程序的探测,eBPF技术能够向下涉及到内核所管理的各种硬件资源,包括CPU、内存、网络、磁盘等,而借助对内核函数的探测,eBPF技术能够向上触及到文件系统、网络子系统等各个内核关键模块。当前受限于内核主线对安全性的要求,eBPF 技术更多的是用于观测,而只在网络子系统中有较大范围的操作能力。

Sched Ext调度类将eBPF的操作能力扩展到调度子系统,而结合eBPF强大的探测能力,打破了调度与各个子系统的信息隔离。首先,对于不同的混部应用,Control Tower调度器提供了足够的自由度允许用户自定义任务的属性,如增加更多的标签,以便于辅助进行决策,其次,对于微服务场景,通过让BPF调度器感知任务之间的调用拓扑关系,有助于进行更高效的调度与服务质量保障。

当前Linux调度器的研发、部署与测试上存在的问题使得调度器研发充满挑战,Control Tower调度器提供了一种解决方案。一方面借助BPF字节码,Control Tower调度器具有极强的可移植性。另一方面,通过将Control Tower调度器封装为容器,也极大方便了部署流程。未来能够与容器编排系统进行结合,提供更方便的使用方式,来使得内核调度器的研发过程与普通应用一样简单。



