\chapter{总结与展望}\label{chap:conclusion}

\section{工作总结}

混部技术是数据中心提升资源利用率的主要手段,但也引发了资源竞争导致应用QoS劣化的问题。本文为解决混部场景下的应用QoS保障问题,首先设计了面向虚拟机的Black Box观测系统,并结合与云厂商的合作经验,选取典型应用进行画像分析。随后,本文设计实现了Control Tower内核调度框架,并展开了针对混部场景的任务调度定制。最后,为解决单点上无法同时允许多种调度机制的问题,本文设计实现了Control Zone,一种面向混部场景的沙箱,支持混部应用与内核调度器的协同部署,从而更好地进行应用QoS保障。具体贡献如下:

\begin{enumerate}

    \item 云场景典型应用监测与画像分析。本文选取云场景中的7种典型应用,并使用KVM虚拟机作为运行环境,分析应用在资源使用倾向与敏感度上的差异。首先,本文基于Prometheus生态设计实现面向KVM虚拟机的Black Box观测系统,能够从Host、Hypervisor与App等多重维度采集的丰富的指标数据。同时,开创性地使用eBPF技术探测虚拟机在系统调用、后端设备模拟上的性能指标。随后,本文开展了基准性能实验与性能劣化实验,并通过华为云内部Benchmark模拟现网负载。最后,基于Black Box观测系统获取典型应用各项性能指标,展开画像分析并给出结论。

    \item 混部场景导向的任务调度定制研究。本文基于画像分析的结论,展开混部场景导向的任务调度定制研究。首先针对应用对于延迟与吞吐量的不同需求,实现了响应度优先与吞吐量优先两种内核配置。随后在Control Tower调度框架的基础上,针对不同混部场景量身定制任务调度机制。本文实现了CPU资源感知与网络资源感知两种Control Tower调度策略,分别能够在SMT硬件场景与LC应用动态负载下更好保障应用的QoS。

    \item 设计实现了一种调度动态可定制沙箱。本文基于KVM虚拟化实现了Control Zone,一种运行时调度可变的沙箱机制。Control Zone提供了一种在单个物理机上同时运行多种不同调度机制的方案。Control Zone允许将混部应用与内核调度策略进行协同部署,能够在特定混部场景下更好地保障应用的QoS。同时,Control Zone支持在运行时改变任务调度策略,从而适应混部场景的软硬件环境变化。最后,Control Zone还支持丰富的资源隔离手段,避免不同KVM虚拟机间的相互干扰。

\end{enumerate}

本文的实验分别在云厂商真实环境与两台实际物理机上展开。在云厂商的真实环境中主要展开典型应用画像实验,使用两台华为云C6型号服务器实例与华为云内部Benchmark来模拟现网环境。两台物理机上的实验主要针对Control Tower调度策略与Control Zone沙箱性能展开,分析了不同内核配置、不同Control Tower调度策略对于特定混部场景的QoS保障效果,以及使用轻量化虚拟机运行时、精简内核的Control Zone在性能上的开销。

本文设计实现的Black Box观测系统、Control Tower任务调度框架与Control Zone调度动态可定制沙箱提供了一种基于内核调度的混部场景QoS保障方案。首先,基于Black Box观测系统采集丰富的指标数据,能够用来分析应用的性能状况。随后,针对不同的混部场景量身定制Control Tower任务调度策略,可以实现更好的应用QoS保障。最后,Control Zone沙箱提供了混部方案的部署与管理手段,能够利用Black Box观测系统验证混部方案的效果并持续迭代。

\section{未来展望}

本文在典型应用画像分析中选择的应用类型有限,但借助已完成的Black Box观测系统,未来可拓展到更多的应用上,如复杂的微服务应用、机器学习应用等。本文针对虚拟机模拟设备后端的性能探测仅涉及vHost net,但当前虚拟机模拟设备十分丰富,未来可进一步拓展到如虚拟块设备上,来获取更丰富的虚拟机性能信息。

本文在Control Tower调度策略上只设计了CPU感知与网络资源感知两种,而实际上eBPF技术能够对大部分内核的运行过程进行探测。如对硬件驱动进行插桩,了解内核中不同硬件资源的分配情况,或对内核其他子系统进行插桩,获取如内存子系统、文件系统等工作状态。其次,通过BPF Map还能够获取用户定义的其他信息,如应用的类型、资源敏感性等。在未来可将这些信息通过Control Tower任务调度框架加入到内核任务调度的定制中,从而更好地在特定混部场景下进行应用的QoS保障。

本文设计的Control Zone沙箱在未来可以跟容器编排系统结合,让内核调度也纳入到容器编排系统的管理中,进一步提升集群调度的灵活性。