\chapter{总结与展望}\label{chap:theories_tech}

\section{工作总结}

混部是当前数据中心常用的手段,但因此也容易引发不同应用之间的资源争用,而为保障应用的QoS,依赖一定的可观测性技术与调度手段。当前以虚拟机技术为代表的沙箱机制在云厂商中占比仍然十分大,而围绕这一场景的可观测性与调度手段通常都需要在内核及Hypervisor层展开。本研究为解决以虚拟机为运行时环境的混部场景下应用QoS保障问题,首先,以虚拟机为目标设计实现了一套可观测性系统,随后,结合与云厂商的合作经验,选取一批典型应用设计实验,并通过可观测性系统采集数据进行画像分析,然后,根据应用画像的结论,针对不同类型应用的混部,进行场景导向的调度定制,最后,设计实现了运行时调度可变的沙箱Control Zone,通过整合前置的研究结果来解决单个服务器上不同混部场景下的QoS保障问题。具体工作如下:

1)面向虚拟机的典型应用监测与画像分析研究。围绕虚拟机在不同层次的特性,基于Promethues生态实现了一套可观测性系统,能够从Host、Hypervisor、App三个层面以虚拟机为粒度采集丰富的指标数据,同时支持在线、离线的数据检索与分析。云场景典型应用画像分析中,选取了华为云竖亥Benchmark中的7种应用,分别从资源使用倾向与干扰敏感度上展开画像分析。

2)混部场景导向的调度定制研究。结合画像分析结论,针对网络型应用于其他类型应用混部,离线应用与其他应用混部的场景,分别设计了响应度优先与吞吐量优先两种内核配置,其次,基于调度队列隔离设计了BPF Scheduler,同时针对不同资源敏感型的任务设计了CPU感知调度策略与网络资源感知调度策略,解决这些原因在混部时的性能保障。

3)运行时调度可变的沙箱研究。数据中心软硬件环境复杂,综合上述分析发现,不同的内核调度配置、不同的调度策略需要在特定的场景下生效,而当前Linux内核不支持同时使能不同的内核调度配置且指提供有限的调度类,同时 Sched Ext调度类也无法加载多个BPF Scheduler。本研究针对如上挑战, 设计实现了Control Zone,一种运行时调度可变的沙箱机制。Control Zone基于虚拟机实现,一方面,利用虚拟机实现了在相同Host上运行不同的Guest配置,同时,各个虚拟机也可加载不同的BPF Scheduler,另一方面,借助虚拟机的强隔离性,提供更多的资源隔离手段来将混部场景限制在一个较为简单的软硬件环境中,从而简化监测及调度的设计难度。

实验部分分别在两台物理机以及云厂商真实环境中展开,在云厂商真实环境中展开典型应用画像实验,了解不同应用之间的差异从而协助挖掘更多的混部组合。两台物理机上的实验则围绕沙箱开销、不同调度配置的内核以及不同BPF调度器展开,一方面说明了沙箱优化策略的效果,另一方面也说明了不同调度配置的内核以及资源感知BPF调度器在混部场景下的有效性。

\section{未来展望}

本研究只设计了CPU与网络两个常用资源的BPF调度实现,实际上,eBPF技术能够提供丰富的内核层次可观测性,借助对驱动程序的探测,eBPF技术能够向下涉及到内核所管理的各种硬件资源,包括CPU、内存、网络、磁盘等,而借助对内核函数的探测, eBPF技术能够向上触及到文件系统、网络子系统等各个内核关键模块。当前首先与主线内核,eBPF技术更多的用于观测,而只在网络子系统中有较大范围的操作能力。Sched Ext调度类将eBPF的操作能力扩展到调度子系统,而结合eBPF强大的探测能力,打破了调度与各个子系统的信息隔离。首先,对于不同的混部应用,BPF调度器提供了足够的自由度允许用户自定义任务的属性,如增加更多的标签,以便于辅助进行决策,其次,对于微服务场景,通过让BPF调度器感知任务之间的调用拓扑关系,有助于进行更高效的调度与服务质量保障。

当前Linux调度器的研发、部署与测试上存在的问题使得调度器研发充满挑战,BPF调度器提供了一种解决方案,借助字节码,BPF调度器天然具有极强的可移植性,通过封装为容器,极大方便了部署流程,而本研究所实现的可观测性系统与Control Zone沙箱机制,能够在BPF调度器及混部方案的测试、部署及验证上提供完善的支持,使得调度器能够与普通应用一样方便地进行迭代开发。