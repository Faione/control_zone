%---------------------------------------------------------------------------%
%->> Frontmatter
%---------------------------------------------------------------------------%
%-
%-> 生成封面
%-

\maketitle% 生成中文封面
\MAKETITLE% 生成英文封面
%-
%-> 作者声明
%-
\makedeclaration% 生成声明页
%-
%-> 中文摘要
%-
\intobmk\chapter*{摘\quad 要}% 显示在书签但不显示在目录
\setcounter{page}{1}% 开始页码
\pagenumbering{Roman}% 页码符号

% 背景 -> 调度是核心解决方式
% 混部场景挑战
% 解决挑战的方式
% 最终效果

混合部署技术(以下简称“混部技术”)是当前数据中心提升资源利用率的主要途径,但也带来了混部应用服务质量(QoS)劣化的问题。云厂商通常与用户协定服务级别目标(SLO)来提供应用QoS的保证,应用QoS劣化不仅会产生SLO的违约赔偿,还会引发用户的流失。实现混部场景下应用QoS保障是云厂商的核心需求,常见的混部场景QoS保障的措施包含QoS劣化监测、资源隔离与任务调度。其中任务调度由于在速度与精度上的优势,是解决单点混部场景QoS保障问题的核心。

混部场景下的任务调度存在三大挑战,(1)混部场景软硬件环境复杂,调度不仅需要满足应用的不同需求,如延迟、吞吐量,还需要需要适应硬件的不同特性,如SMT、NUMA。(2)混部场景中应用负载动态变化,如服务型应用的负载随时间波动,调度需要及时感知并进行调整。(3)服务器资源丰富,能够同时运行大量应用,这些应用构成多样的混部场景,而单一的任务调度难以适配所有场景。

为解决上述挑战,本文(1)针对云场景中的7种典型应用展开画像分析。首先设计实现黑匣子(Black Box),一种面向KVM虚拟机的多维观测系统。随后展开实验,分析应用在资源使用倾向、敏感度上的差异。(2)从调度子系统配置与内核任务调度定制两方面展开混部场景导向的任务调度研究。在调度子系统配置上,本文设计响应度优先与吞吐量优先配置,分别实现混部应用99分位尾延迟降低最高39.2\%与执行速度提升最高1.6 $\times$的效果。在内核调度策略定制上,本文设计塔台(Control Tower)内核任务调度框架,用于针对混部场景定制内核任务调度策略来保障应用的QoS,并实现CPU感知与网络感知调度策略。其中CPU感知调度策略实现混部应用99分位尾延迟降低最高90.4\%的效果,并在超线程场景下仍有最高71.2\%的效果。(3)设计实现受控空域(Control Zone),一种面向混部场景的调度动态可定制沙箱。Control Zone基于KVM虚拟化技术实现,提供了一种在单台物理机上同时运行多种调度机制的方案,并支持混部应用与Control Tower调度策略的协同部署。同时,Control Zone支持在运行时切换内核调度策略以适应混部场景软硬件环境的变化。在性能上,Control Zone采用轻量化虚拟机监视器与精简内核,沙箱启动耗时在370ms内,且即使存在虚拟化开销,仍实现部分应用99分位尾延时降低最高38.5\%的效果。

\keywords{数据中心,混合部署,内核调度,eBPF,服务质量}% 中文关键词
%-
%-> 英文摘要
%-
\intobmk\chapter*{Abstract}% 显示在书签但不显示在目录

Co-Location technology (hereinafter referred to as "Co-Location") is a primary approach for enhancing resource utilization in current data centers. However, it also brings about the issue of quality of service (QoS) degradation for applications under hybrid deployment. Cloud providers usually establish service level objectives (SLOs) with users to guarantee application QoS. Degradation in application QoS can result not only in SLO violation penalties but also in user attrition. Ensuring application QoS in hybrid deployment scenarios is a core requirement for cloud providers. Common measures to guarantee QoS in hybrid deployment scenarios include QoS degradation monitoring, resource isolation, and task scheduling. Among these, task scheduling is crucial for addressing single-point QoS assurance problems in hybrid scenarios due to its advantages in speed and precision.

There are three major challenges in task scheduling for hybrid scenarios: (1) The complex software and hardware environment of hybrid scenarios requires scheduling to meet various application needs, such as latency and throughput, while also adapting to different hardware characteristics, such as SMT and NUMA. (2) Application loads in hybrid scenarios are dynamically changing; for example, the load of service applications fluctuates over time, requiring the scheduling to promptly sense and adjust to these changes. (3) Servers are resource-rich, capable of running numerous applications simultaneously, creating diverse hybrid scenarios, and a single task scheduling approach cannot cater to all scenarios.

To address these challenges, this paper (1) conducts a profiling analysis of seven typical applications in cloud scenarios. First, it designs and implements Black Box, a multidimensional observation system for KVM virtual machines. Subsequently, it performs experiments to analyze differences in resource usage tendencies and sensitivity among applications. (2) It explores hybrid scenario-oriented task scheduling from two aspects: scheduler subsystem configuration and kernel task scheduling customization. In scheduler subsystem configuration, it designs responsiveness-prioritized and throughput-prioritized configurations, achieving up to 39.2\% reduction in 99th percentile tail latency and up to 1.6 $\times$ improvement in execution speed for hybrid applications, respectively. In kernel scheduling strategy customization, it designs the Control Tower kernel task scheduling framework to customize kernel task scheduling strategies for hybrid scenarios to ensure application QoS, implementing CPU-aware and network-aware scheduling strategies. The CPU-aware scheduling strategy achieves up to 90.4\% reduction in 99th percentile tail latency for hybrid applications and still shows up to 71.2\% improvement under hyperthreading scenarios. (3) It designs and implements Control Zone, a dynamically customizable scheduling sandbox for hybrid scenarios. Control Zone, based on KVM virtualization technology, provides a scheme for running multiple scheduling mechanisms on a single physical machine and supports the collaborative deployment of hybrid applications and Control Tower scheduling strategies. Additionally, Control Zone supports runtime switching of kernel scheduling strategies to adapt to changes in the software and hardware environment of hybrid scenarios. Performance-wise, Control Zone employs a lightweight virtual machine monitor and a streamlined kernel, with sandbox startup time within 370ms, and despite the virtualization overhead, it achieves up to 38.5\% reduction in 99th percentile tail latency for some applications.


\KEYWORDS{Data center, Co-Location, Kernel Scheduling, eBPF, Quality of Service}% 英文关键词

\pagestyle{enfrontmatterstyle}%
\cleardoublepage\pagestyle{frontmatterstyle}%

%---------------------------------------------------------------------------%
