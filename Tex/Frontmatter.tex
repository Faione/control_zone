%---------------------------------------------------------------------------%
%->> Frontmatter
%---------------------------------------------------------------------------%
%-
%-> 生成封面
%-

\maketitle% 生成中文封面
\MAKETITLE% 生成英文封面
%-
%-> 作者声明
%-
\makedeclaration% 生成声明页
%-
%-> 中文摘要
%-
\intobmk\chapter*{摘\quad 要}% 显示在书签但不显示在目录
\setcounter{page}{1}% 开始页码
\pagenumbering{Roman}% 页码符号

% 背景 -> 调度是核心解决方式
% 混部场景挑战
% 解决挑战的方式
% 最终效果

混合部署技术(以下简称“混部技术”)是当前数据中心提升资源利用率的主要途径,但也带来了混部应用服务质量(QoS)劣化的问题。云厂商通常与用户协定服务级别目标(SLO)来提供应用QoS的保证,应用QoS劣化不仅会产生SLO的违约赔偿,还会引发用户的流失。实现混部场景下应用QoS保障是云厂商的核心需求,常见的混部场景QoS保障的措施包含QoS劣化监测、资源隔离与任务调度。其中任务调度由于在速度与精度上的优势,是解决单服务器上混部场景QoS保障问题的核心。

混部场景下的任务调度存在三大挑战,(1)混部场景软硬件环境复杂,调度不仅需要满足应用的不同需求,如延迟、吞吐量,还需要需要适应硬件的不同特性,如SMT、NUMA。(2)混部场景中应用负载动态变化,如服务型应用的负载随时间波动,调度需要及时感知并进行调整。(3)服务器资源丰富,能够同时运行大量应用,这些应用构成多样的混部场景,而单一的任务调度难以适配所有场景。

为解决上述挑战,本文(1)针对云场景中的7种典型应用展开观测与画像分析。首先设计实现黑匣子(Black Box),一种面向KVM虚拟机的多维观测系统。随后展开基准与干扰实验,分析典型应用在资源使用倾向、敏感度上的差异。(2)从调度子系统配置与内核任务调度定制两方面展开混部场景导向的任务调度研究。针对调度子系统配置设计响应度优先与吞吐量优先配置,分别实现混部应用99分位尾延迟降低最高39.2\%与执行速度提升最高1.6 $\times$的效果。针对内核调度策略定制设计塔台(Control Tower)内核任务调度框架,用来为混部场景定制内核任务调度策略来保障应用的QoS,并实现CPU感知与网络感知两种调度策略。其中CPU感知调度策略实现混部应用99分位尾延迟降低最高90.4\%的效果,并在超线程场景下仍有最高71.2\%的效果。(3)设计实现受控空域(Control Zone),一种面向混部场景的调度动态可定制沙箱。Control Zone基于KVM虚拟化技术实现,提供了一种在单台服务器上同时运行多种调度机制的方案,并支持混部应用与Control Tower调度策略的协同部署。同时,Control Zone支持在运行时切换内核调度策略以适应混部场景软硬件环境的变化。在性能上,Control Zone采用轻量化虚拟机监视器与精简内核,沙箱启动耗时在370ms内,且即使存在虚拟化开销,仍实现部分应用99分位尾延时降低最高38.5\%的效果。

\keywords{数据中心,混合部署,内核调度,eBPF,服务质量}% 中文关键词
%-
%-> 英文摘要
%-
\intobmk\chapter*{Abstract}% 显示在书签但不显示在目录

Co-location technology is currently the main method to improve resource utilization in data centers. However, it also brings about the issue of degradation in the quality of service (QoS) for mixed applications. Cloud providers typically establish service level objectives (SLOs) with users to ensure application QoS. Degradation in application QoS can result not only in penalties for violating SLOs but also in user attrition. Ensuring QoS in hybrid scenarios is a core requirement for cloud providers. Common measures to guarantee QoS in hybrid scenarios include QoS degradation monitoring, resource isolation, and task scheduling. Among these, task scheduling is crucial for solving QoS assurance issues in single-server hybrid scenarios due to its speed and accuracy advantages.

There are three main challenges in task scheduling in hybrid scenarios: (1) The complex software and hardware environment in hybrid scenarios requires scheduling to meet different application needs, such as latency and throughput, while also adapting to various hardware characteristics like SMT and NUMA. (2) The dynamic changes in application load in hybrid scenarios, such as the time-varying load of service applications, necessitate timely perception and adjustment in scheduling. (3) Servers have abundant resources capable of running numerous applications simultaneously, creating diverse hybrid scenarios that a single task scheduling strategy cannot accommodate.

To address these challenges, this paper (1) conducts profiling analysis of seven typical applications in cloud scenarios. First, we design and implement Black Box, a multidimensional observation system for KVM virtual machines. We then conduct experiments to analyze the differences in resource usage preferences and sensitivity among applications. (2) Research task scheduling in hybrid scenarios from two aspects: scheduling subsystem configuration and kernel task scheduling customization. For scheduling subsystem configuration, we design responsiveness-first and throughput-first configurations, achieving up to 39.2\% reduction in the 99th percentile tail latency and up to 1.6 $\times$ improvement in execution speed for hybrid applications, respectively. For kernel scheduling strategy customization, we design the Control Tower kernel task scheduling framework, which customizes kernel task scheduling strategies to ensure application QoS in hybrid scenarios. This includes CPU-aware and network-aware scheduling strategies. The CPU-aware scheduling strategy achieves up to 90.4\% reduction in the 99th percentile tail latency for hybrid applications, with up to 71.2\% reduction even in hyperthreading scenarios. (3) We design and implement Control Zone, a dynamically customizable sandbox for scheduling in hybrid scenarios. Based on KVM virtualization technology, Control Zone offers a solution for running multiple scheduling mechanisms on a single server and supports the collaborative deployment of hybrid applications with Control Tower scheduling strategies. Control Zone also supports switching kernel scheduling strategies at runtime to adapt to changes in the software and hardware environment of hybrid scenarios. In terms of performance, Control Zone employs a lightweight virtual machine monitor and a streamlined kernel, with sandbox startup taking less than 370ms. Despite virtualization overhead, it achieves up to 38.5\% reduction in the 99th percentile tail latency for some applications.

\KEYWORDS{Data center, Co-Location, Kernel Scheduling, eBPF, Quality of Service}% 英文关键词

\pagestyle{enfrontmatterstyle}%
\cleardoublepage\pagestyle{frontmatterstyle}%

%---------------------------------------------------------------------------%
