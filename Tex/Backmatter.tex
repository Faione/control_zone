%---------------------------------------------------------------------------%
%->> Backmatter
%---------------------------------------------------------------------------%
\chapter[致谢]{致\quad 谢}\chaptermark{致\quad 谢}% syntax: \chapter[目录]{标题}\chaptermark{页眉}
%\thispagestyle{noheaderstyle}% 如果需要移除当前页的页眉
%\pagestyle{noheaderstyle}% 如果需要移除整章的页眉

三年研究生生涯如白驹过隙,转瞬间又到了毕业的时刻,此刻回首,一路都是老师、朋友和家人的身影。

首先,我要感谢我的导师王卅老师。在三年的研究生生涯中,无论是项目攻关的关键时刻,还是毕业论文撰写的紧要期间,卅老师总会及时出现,并给出针对性意见协助解决问题,对于我个人,卅老师是无微不至的朋友,帮助我解决研究上、生活上的难题,也是学习的榜样,教会我为人处世的道理和坚持不懈的科研精神。

其次,我要感谢姚老师、唐老师以及信息高铁团队的所有老师同学们。那段一起奋斗攻坚的时间,历练了我的工程能力,也让我学会如何在团队中相互合作。

当然,我要感谢课题组的同学们,同门杜哥协助我攻坚了许多项目,子豪师兄在就业上为我提供了很多指导,紫微师姐在学术研究上带领我入门,家祺学弟在毕业论文上的协助。以及敬远、河阳和甄好老师三位答辩秘书在毕设流程上的帮助。

最后要感谢我的家人们,求学之路上家人永远最坚实的依靠,你们的付出才有了我的今天。


\rightline{2023年6月}
\chapter{作者简历及攻读学位期间发表的学术论文与其他相关学术成果}

\section*{作者简历:}

2017年9月——2021年6月,在西南财经大学大学经济信息工程学院(系)获得学士学位。

2021年9月——2024年6月,在中国科学院计算技术研究所攻读硕士学位。

\section*{参加的研究项目:}

\begin{enumerate}
    \item 江苏省重大科研设施预研《信息高铁综合试验基础设施——算力网》,2022.04-2023.06
    \item 华为合作项目:《软硬件协同细粒度QoS保障机制研究》,2023.06-2024.06
\end{enumerate}


\cleardoublepage[plain]% 让文档总是结束于偶数页,可根据需要设定页眉页脚样式,如 [noheaderstyle]
%---------------------------------------------------------------------------%
