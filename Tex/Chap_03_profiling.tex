\chapter{典型应用观测与画像分析}\label{chap:profiling}

\section{云场景典型应用选择}

% 软件环境复杂 -> 应用需求不同 -> 软件架构复杂 -> 存在共性的基础应用 -> 基础应用的性能是整体性能的关键 -> 选取典型应用进行分析

数据中心软件环境复杂,同时托管的应用种类较为丰富,为画像分析带来了许多挑战。首先,数据中需要根据用户的使用需求,提供不同的软件环境,如针对虚拟机,需要提供虚拟化环境,而对于容器,则需要提供如K8s等,本研究结合与云厂商的合作经验,选择云场景中使用较普遍的虚拟机作为目标场景,在该场景中,应用的运行环境由宿主机系统、虚拟化运行时及来宾操作系统共同完成。其次,数据中心托管应用丰富的主要原因是需求的差异,然而当前随软件架构的不断进步及开源、云原生等理念的推广,应用在设计时会使用到许多相同的公共组件,如数据库、消息中间件等,这些公共组件构成了现代软件的基石,同时也是应用性能的关键,因此在本研究中,结合与云厂商的合作经验,从传统数据库、键值存储、消息中间件、大数据、媒体处理等领域中选择了如表~\ref{tab:typical_application}所示的7种典型应用进行画像分析。

\begin{table}
    \bicaption{\quad 云场景典型应用}{\quad Typical Applications in Cloud Scenarios}% caption
    \label{tab:typical_application}
    \footnotesize% fontsize
    \setlength{\tabcolsep}{4pt}% column separation
    \renewcommand{\arraystretch}{1.5}% row space 
    \centering
    \begin{tabular}{lc}
        \hline
        %\multicolumn{num_of_cols_to_merge}{alignment}{contents} \\
        %\cline{i-j}% partial hline from column i to column j
        类型 & 应用\\
        \hline
        SQL & mysql\\
        NoSQL & elasticsearch\\
        Web Server & nginx\\
        K-V Store & redis、memcached\\
        Message Queue & kafka\\
        Media & render\\
        \hline
    \end{tabular}
\end{table}

这些应用在领域上的差异,决定了其在功能实现上的不同,并会使用到不同的资源,因此各自对调度有不同的需求:

1)数据库类型的应用,如Mysql,传统的关系型数据库提供了完善了数据存储、检索查询及事务管理等功能,实现中原始数据通常保存在磁盘上,在进行大量数据查询时,不仅会使用到较多的CPU资源,同时也会占用相当的IO资源,因此需要调度器保障其CPU资源,并避免IO上的干扰。

2)键值存储类应用,如Redis、Memcached,通常作为信息缓存服务,提供快速的数据存储与查询,由于数据存储在内存中,因此对内存资源存在较高要求,对调度的需求主要体现在两方面,一方面需要足够的内存资源,另一方面则需要避免在内存带宽上的干扰。

3)网络服务类型应用,如Nginx,网络页面通常都具有交互属性,而延迟会极大影响用户的使用体验,因此这类应用要求调度能够在网络请求到达时,及时的为网络服务分配CPU,以尽早地对连接进行处理。

4)媒体处理应用,如ffmjpeg,通常是离线应用,运行时会占用大量的CPU资源,来进行编解码操作,这类应用需要调度分配足够长的时间片,同时希望能减少上下文地切换,以便更好地利用局部性来加快执行速度。

\section{观测系统设计与指标收集}

\subsection{面向虚拟机的多维指标采集设计}

% 分解虚拟机: Host侧 -> Hypervisor侧 -> 应用侧
% 分解采集系统:
% - 采集侧
% - 查询、存储、分析侧

本研究中使用KVM虚拟机作为混部应用基本软件环境。而结合第二章中对于虚拟化沙箱技术的分析,本研究设计了一种如图~\ref{}所示的面向虚拟机的多维度指标采集机制。

首先,在宿主机上,KVM虚拟机由多个进程组成,因此在这一维度,可以利用宿主机Kernel提供的多种进程监测机制,如/proc内核文件系统中记录的进程信息,以及cgroup子系统中对进程的记账信息,来评测虚拟机进程的性能。其次,虚拟机由各个模拟硬件组成,这些设备由Hypervisor进行模拟,在此维度,能够利用Hypervisor提供的丰富接口,来获取虚拟机的各个虚拟机设备的性能信息,以vCPU为例,本研究中KVM内核模块负责虚拟机的vCPU模拟,而KVM内核模块通过debug文件系统提供了丰富的vCPU监测信息,如包括IO、挂起、中断等在内的陷出计数,以及MMU缓存缺失计数等。最后,虚拟化环境最终需要为应用的运行提供服务,应用级指标能够体现出应用的真实性能,而针对这一维度的指标采集,主要从两个方面入手,对于Client-Server类的服务型应用,可以在外部通过Overlay网络建立对延迟、吞吐量的追踪,部分应用也内置了观测接口,提供性能信息的采集。

而为更深入地探测虚拟机性能,本研究还基于eBPF设计了一种在内核侧监测的机制。首先,在宿主机维度,进程对于硬件资源的使用通常都需要通过系统调用完成, 如使用网络资源时,需要使用send或recv系统调用,使用IO资源时,需要进行read、write系统调用,对虚拟机而言,这种方式能够监测陷出到用户态之后的设备模拟过程。其次,在Hypervisor维度,为追求设备模拟的效率,一些模拟硬件的后端实现并不要求虚拟机陷出到用户态,而直接在内核态完成整个模拟过程,如图~\ref{fig:vhost_net}所示,vHost net作为一种内核态实现的模拟网络设备后端,其网络发送的过程就直接在内核中通过函数调用进行,而这些函数的执行状态能够反映出模拟网络设备后端的性能。

\begin{figure}[!htbp]
    \centering
    \includegraphics[width=0.7\textwidth]{vhost_net}
    \bicaption{\quad vHost Net模拟网络设备后端}{\quad vHost Net Simulate network device backend}
    \label{fig:vhost_net}
\end{figure}

综合以上分析,本研究设计了一种实时指标采集分析系统,能从上述维度以虚拟机为粒度机进行指标采集,同时能够提供指标的存储、查询与分析服务。系统中定义了Observer与Observed两种角色,首先,Observer与Observed以统一的格式对指标进行编解码,随后,Observed中部署监控组件以对外部提供监控服务,最后,Observer周期性地轮询各个Observed,并将收集到的原始指标按标签进行区分存储,并根据预定义的规则对指标进行聚合处理,同时对外提供查询与分析服务。系统中可以有多个Observer来分摊采集压力,并且每个节点可以按需要同时承担Observer与Observed两种角色。

\subsection{实时可观测性系统实现}

% Observer实现
% - Prometheus、Grafana、Collector Utils
% Observed实现
% - Exporter的实现细节
%   - Ebpf Exporter
%       - Syscall 追踪
%       - vHost Net 追踪

\begin{figure}[!htbp]
    \centering
    \includegraphics[width=0.8\textwidth]{monitor_arch}
    \bicaption{\quad 采集系统架构}{\quad Metric Collection System Architecture}
    \label{fig:monitor_arch}
\end{figure}

本研究结合云原生的实践经验,选择基于Prometheus来实现可观测性系统。Prometheus\citep{prometheus}是云厂商中广泛使用的一种指标监测系统,具有可扩展性强、灵活性好、生态丰富、部署便捷等特点,Prometheus的核心贡献在于提供了一种扩展性强的时序数据结构Metric以及围绕Metric的PromQL向量查询语言、向量数据存储方式以及编解码机制,在此基础上,Prometheus还实现了一套采集系统,提供了Promethues Server、Grafana及Node Exporter等核心组件,由于Promethues完全开源并提供了各种语言的Metric采集基础库,因此吸引了大量开发者及云厂商的参与,催生出丰富的组件生态。当前Prometheus已经成为指标采集的主要标准,并随DevOps的发展被越来越广泛地使用。

基于Prometheus生态的实时可观测性系统架构如图~\ref{fig:monitor_arch}所示,其中Observer的基于Prometheus Server与Grafana标准组件实现,在向量数据存储上,将原生的TiDB替换位InfluxDB,InfluDB使用以Rust语言编写,性能更强、扩展性更好。在采集上,Observer中虚拟机各维度数据的采集组织为一个服务发现配置文件,基于Prometheus的服务发现规则编写。而在数据聚合上,各维度原始数据首先需要进行初步处理以生成有效数据,Observer提供了PromQL编写了基本的聚合规则,实现对于标量数据的变化速率、99分位数据等的统计,同时这些规则会转化为Grafana中的Dashboard以进行实时监测。

可观测性系统中Observed的实现则包含了各个维度的Exporter的开发与配置。如图~\ref{fig:exporters}所示,Observed中包含了6种Exporter,能够在Host、Hypervisor、App三个维度围绕虚拟机提供丰富的数据采集,原始指标如表~\ref{tab:metric_list}所示,包含CPU、Cache、Memroy、IO、Network等各个子系统,总数量达到344个。

\begin{figure}[!htbp]
    \centering
    \includegraphics[width=0.9\textwidth]{exporters}
    \bicaption{\quad Exporters设计}{\quad Exporters Design}
    \label{fig:exporters} 
\end{figure}

1)Node Exporter是Prometheus提供的基础监控组件,提供Host维度的基础指标采集。其实现中主要利用了Linux Kernel提供的交互结果,来获取Host各个子系统中的运行状态信息,如通过/proc内核文件系统,获取运行时间、内存信息等。Observed实现中使能了NodeExporter的Perf Event监测模块,模块支持采集Host上的性能事件,包括cycles、instruction及cache miss等。

2)Resctrl Exporter实现了Resctrl子系统的指标采集。Intel RDT等硬件技术提供了进程粒度的CPU末级缓存与内存带宽的监测能力,Linux Resctrl作为这类技术的软件接口,提供了一个内核文件系统,用户可以通过在此文件系统中创建文件夹的形式来创建Monitor Group,而将进程加入到其中之后,Resctrl子系统就会开始为进程记录LLC与内存带宽信息。Resctrl Exporter基于Resctrl子系统实现, 通过扫描子系统中的Monitor Group来生成相关的指标信息,并将Monitor Group作为标签以便于后续聚合。

3)KVM Exporter实现了Hypervisor维度的指标采集。KVM内核模块提供了一个Debug文件系统,在最顶层的目录中,记录了内核模块的运行信息,如虚拟机陷出的总计数等,同时,对于每个虚拟机进程,还会创建对应的子目录,并以单个虚拟机为粒度记录内核模块的运行数据,最后,在每个虚拟机目录中,还会为每个vCPU创建一个目录,记录vCPU的时钟源相关信息。KVM Exporter基于上述机制实现,通过扫描Debug系统中的目录,并为遍历每个虚拟机条目来生成相关的指标信息,默认情况下使用虚拟机进程号作为标签,也允许传入一个配置文件,来实现虚拟机进程号与虚拟机名称之间的转化。

4)Libvirt Exporter是社区开发的一个监控组件,提供了Hypervisor维度的指标采集。Libvirt是一个统一的虚拟化接口层,提供了丰富的虚拟机管理选项,Libvirt Exporter基于Libvirt Monitor接口实现,能够采集虚拟机的各个设备的性能指标,如vCPU时间片、内存使用量、网络设备的吞吐量及IO设备的吞吐量等。Observed实现中为Libvirt Exporter增加的Perf Event监测功能,从而能够提供虚拟机粒度的性能事件指标信息,如cycles、instruction及cache miss等。

\begin{figure}[!htbp]
    \centering
    \includegraphics[width=0.7\textwidth]{syscall_hook}
    \bicaption{\quad eBPF Syscall插桩}{\quad eBPF Syscall Hooking}
    \label{fig:syscall_hook}
\end{figure}

5)eBPF Exporter是社区开发的一个监控组件, 提供了内核侧的自定义指标采集。eBPF Exporter能够加载用户编译为字节码的eBPF程序,并通过RingBuffer收集内核侧eBPF程序采集的数据转化为指标。Observed中实现了系统调用及vHost Net内核模块两个eBPF采集程序。内核中系统调用分布在各个子模块中,并通过注册系统调用的方式加入到系统调用表中,用户态应用发起的系统调用会触发一次中断,并最终进入系统调用分发函数,因此可以利用raw tracepoint机制在系统调用分发函数的开始与结束处进行eBPF插桩,如图~\ref{fig:syscall_hook}所示,每当进程进入系统调用中时,首先可以通过BPF Helper函数获取当前task结构体指针,并读取进程号等标识信息,其次通过读取系统调用上下文中的ax寄存器,来获取当前的系统调用号,这两个信息提供了基本的指标标签,基同时,利用BPF Stack来记录时间戳信息,就能够在系统调用返回时计算整个系统调用的执行时间,从而实现对于而对进程每个系统调用吞吐、延迟的监测。vHost Net内核模块的监测则围绕worker内核线程展开,如图~\ref{fig:vhost_net}所示,worker线程中可能执行4种不同的回调函数,分别用于虚拟机的网络发送与接受,通fentry机制过在这些函数的入口与出口出进行插桩,就能够统计出各个函数执行频率以及延时等信息。

\begin{table}
    \bicaption{\quad 指标列表}{\quad Metric list}% caption
    \label{tab:metric_list}
    \footnotesize% fontsize
    \setlength{\tabcolsep}{4pt}% column separation
    \renewcommand{\arraystretch}{1.5}% ro w space 
    \centering
    \begin{tabular}{lcc}
        \hline
        %\multicolumn{num_of_cols_to_merge}{alignment}{contents} \\
        %\cline{i-j}% partial hline from column i to column j
        层级 & 来源 & 指标\\
        \hline
        Host & Kernel(271) & [cpu] sys\_time、user\_time、freq ... \\
        & & [mem] usage、avail、swapfree、bounce、hugepage ...\\
        & & [network] transmit、netstat、sockstat、ip、speed ...\\
        & & [disk]flush requests、flush request time、read、write ...\\
        & & ...\\
        & eBPF(2) & [syscall] count、duration\\
        & Resctrl(3) & llc cap、mem b/w local、mem b/w total\\
        Hypervisor & Libvirt(68) & [vcpu] sys\_time、user\_time、wait、delay...\\
        & & [mem] usable、avail、dick cache、rss ...\\
        & & [perf] cycle、instruction、cache miss ...\\
        & & [block] flush request、flush time、read、write ...\\
        & & [interface] receive、transmit ...\\
        & & ...\\
        & KVM(59) & vm\_exit、io\_exit、irq\_exit、irq\_inject、halt\_poll ...\\
        App & Log/Envoy & qos, latency, rate, score ...\\
        \hline
    \end{tabular}
\end{table}

\section{实验设计}

\subsection{基准性能实验}

实验中使用两台C6服务器作为Client和Server,服务器具体配置如表~\ref{tab:c6_info}所示,其中Client服务器额外承担实验管理的功能,同时作为Observer角色部署Prometheus Server等组件,而Server主要承载具体的应用,同时作为Observed角色部署上述设计实现的6种Exporter,并为Exporter设置Numa亲和性以避免对应用产生干扰。基准性能实验测试无干扰情况下各个应用的基础性能,对于每个典型应用,使用华为云提供的竖亥Benchmark中的模拟负载多样的负载来监测不同场景下的基准性能。

\begin{table}
    \bicaption{\quad C6服务器信息}{\quad C6 Information}% caption
    \label{tab:c6_info}
    \footnotesize% fontsize
    \setlength{\tabcolsep}{4pt}% column separation
    \renewcommand{\arraystretch}{1.5}% row space 
    \centering
    \begin{tabular}{lc}
        \hline
        类型 & 信息 \\
        \hline
        OS & Ubuntu 22.04 ( Kernel 5.15.0) \\
        CPU & Intel Xeon Gold 6151 (18 cores) * 2 \\
        Processor Core Frequency & 3GHz,Turbo 3.4GHz \\
        L1 Caches & 32KB *,  8-way set associative, split D/I \\
        L2 Caches & 1024KB, 16-way set associative \\
        L3 Caches & 25344KB, 11-way set associative \\
        Main Memory & 32GB * 12, 2666MHz DDR4 \\
        Storage & Avago MegaRAID, 2.18T SAS RAID0, 87.322T SATA RAID0 \\
        NIC & Intel Corporation Ethernet Connection X722 for 10GbE SFP+(10Gbit) \\
        \hline
    \end{tabular}
\end{table}

% 1)Redis、Memcached。使用memtier_benchmark来产生负载,负载的模型由华为云提供,包括Normal、Random两类模拟负载

% 2)Mysql。使用TPCC作为基准实验负载

% 3)Elasticsearch。使用YCSB作为基准负载

% 4)Kafka。

% 5)Nginx。

% 6)Render。使用华为内部的模拟负载。

\subsection{性能劣化实验}

% 干扰应用
% - 干扰有效性
% - 噪声控制

性能劣化实验通过增加目标应用与干扰应用的混部场景,分析应用对于不同干扰类型的敏感程度。在干扰应用的选择上,使用stress-ng来产生不同类型的干扰。stress-ng是Linux系统中常用的压力测试工具,能够模拟各种资源的压力场景,而实验中则利用其模拟各种压力场景的特性来制造干扰,并观测目标应用的性能劣化情况。stress-ng提供了丰富的配置参数,本课题选用如表~\ref{tab:arg_list}所示的参数配置,来制造CPU、Cache、内存、IO、Network上的干扰。

\begin{table}
    \bicaption{\quad stress\_ng 参数列表}{\quad stress\_ng args list}% caption
    \label{tab:arg_list}
    \footnotesize% fontsize
    \setlength{\tabcolsep}{4pt}% column separation
    \renewcommand{\arraystretch}{1.5}% row space 
    \centering
    \begin{tabular}{lcc}
        \hline
        %\multicolumn{num_of_cols_to_merge}{alignment}{contents} \\
        %\cline{i-j}% partial hline from column i to column j
        子系统 & 参数 & 说明\\
        \hline
        CPU	    & --cpu & 循环执行sqrt(rand())的线程数量\\
	            & --cpu-load & 线程负载的比率\\
        Cache	& --cache & cache抖动线程数量\\
	            & --cache-level	&测试指定等级的Cache\\
	            & --icache	&指令cache抖动的线程数量\\
        IO	    & --io	&循环执行sync()的线程数量\\
	            & --iomix	&执行混合I/O操作的线程数量\\
	            & --hdd	&循环执行write()/unlink()的线程数量\\
	            & --seek	&执行随机seek I/O的线程数量\\
        Memory	& --vm	&循环执行匿名mmap的线程数量\\
	            & --vm-bytes	&执行vm操作的buffer大小\\
	            & --memrate	&执行read/writes的线程数量\\
	            & --memrate-bytes	&执行内存操作的buffer大小\\
	            & --malloc	&执行malloc/realloc/free的线程数量\\
	            & --memcpy	&执行memory copy的线程数量\\
        Network	& --sock	&执行Socket I/O的线程数量\\
	            & --epoll	&执行epoll处理的线程数量\\
        \hline
    \end{tabular}
\end{table}

\section{画像分析与结论}

\subsection{典型应用资源使用倾向分析}

\subsection{典型应用资源敏感度分析}

\section{本章小结}

本章主要论述了对应用进行画像分析的过程,首先结合与云厂商的合作经验,确定了以虚拟机为沙箱环境运行优先应用的场景,并选取了云场景中常见的8种应用作为画像分析的对象。

然后对场景进行拆解分析,并提出了一种从Host、Hypervisor到App三个层次的协同监控机制设计,利用虚拟机本身作为进程的特殊性,串联三个维度的指标,并创新性地提出了一种基于eBPF对虚拟机监测的方式,对于本身为进程的虚拟机,监测其在系统调用上的吞吐与延时,而针对虚拟机的特殊性,则通过在虚拟化设备后端的关键路径上进行监测,实现更丰富的指标采集。

随后根据上述设计,设计实现了一套可观测性系统,除利用到开源的Promethues、Node Exporter外,还根据监控需要,实现了Libvirt Exporter、KVM Exporetr、Resctrl Exporter以及Kernel Exporter,完成所有的设计需求。

在可观测性基础设施实现完毕后,本章继续讨论实验的设计,实验围绕8种常见应用展开,并设计了有干扰实验与无干扰实验,在有干扰实验中,利用资源限制手段来减少干扰中的噪声,通过制造CPU、Cache、Memory及IO四种干扰,来分析应用对于干扰的敏感程度,而通过无干扰实验,来分析应用的资源使用倾向

最后在所有实验完成之后,分析从Prometheus中采集得到数据,来为应用进行画像分析,提供了应用的资源使用倾向及干扰敏感度评级,并探讨了除LC与BE外更多的混部机会。